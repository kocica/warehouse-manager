Tato kapitola se zabývá návrhem nástrojů potřebných pro optimalizaci skladových operací. Výsledné řešení se bude skládat z pěti \uv{kooperujících} nástrojů. Veškeré nástroje budou implementovány v \texttt{C++} a pro grafickou nástavbu bude použit framework \texttt{Qt} verze $5$.

\begin{itemize}
    \item \textbf{Generátor produkčních dat} -- Vzhledem k nemožnosti použití zákaznických objednávek za účelem udržení obchodního tajemství je potřeba vytvořit generátor dat založený na jistém matematickém modelu, aby data nebyla čistě náhodná, nýbrž spolu korelovala.
    \item \textbf{Simulátor skladu} -- Nástroj, který na základě vygenerovaných objednávek a~vytvořeného layoutu skladu dokáže odsimulovat pickování vygenerovaných objednávek a~poskytnout statistiky, zejména pak dobu potřebnou k pickování a~zatížení lokací a~dopravníků.
    \item \textbf{Pathfinder} (česky hledač cest) -- Nástroj, který dokáže nalézt optimální cestu objednávky skrze sklad.
    \item \textbf{Optimalizátor rozložení produktů} -- Nástroj, který dokáže optimalizovat alokaci jednotlivých produktů do lokací skladu za účelem snížení doby potřebné k napickování veškerých zákaznických objednávek.
    \item \textbf{Warehouse Manager} (česky skladový manažer) -- Grafická aplikace, která uživateli umožní vytvořit model skladu. Dále bude také obsahovat všechny čtyři zmíněné nástroje v grafické podobě.
\end{itemize}

\section{Generátor produkčních dat}
\label{navrhGenerator}
Generátor bude mít za úkol generovat dvě sady syntetických objednávek. Jedna sada bude určená pro trénování genetického algoritmu (reprezentující historické data společnosti) a~druhá pro vyhodnocení natrénovaného modelu (reprezentující \uv{budoucí} data společnosti). Tyto dvě sady spolu musí korelovat, a proto bude vytvořen jeden matematický model (pravděpodobnosti jednotlivých produktů vypočteny na základě ADU -- \emph{Average daily units} a množství na základě ADQ -- \emph{Average daily quantity}). Pro generování ADU i ADQ pro jednotlivé produkty se využije Normální (Gaussovo) rozdělení.

\section{Simulátor skladu}
\label{navrhSimulator}
Bude implementován pomocí vhodné knihovny umožňující simulaci za pomoci diskrétních událostí. Zejména bude sloužit pro vyhodnocení kvality jednotlivých řešení v optimalizačním nástroji, ale bude také možné jeho samostatné využití např. pro identifikaci úzkých míst či získání statistik zpracování objednávek ve skladu.

\section{Pathfinder}
\label{navrhPathfinder}
Tento nástroj bude sloužit pro nalezení optimální cesty objednávky skladem, tak, aby urazila co nejmenší možnost vzdálenost. Toho bude dosaženo pomocí mravenčího algoritmu popsaném v \ref{sec:mmas}.

\section{Optimalizátor rozložení produktů}
\label{navrhOptimalizator}
Řešený optimalizační problém (alokace produktů do slotů lokací) má velmi mnoho společného s problémem obchodního cestujícího. Stejně jako obchodní cestující musí projít všechna města, je zde potřeba alokovat všechny produkty do slotů. A stejně tak jako obchodní cestující nesmí navštívit žádné město vícekrát, ani žádný z produktů nelze alokovat do dvou či více slotů zároveň. Budou tedy implementovány čtyři evoluční algoritmy, jež budou implementovány podle redefinic jednotlivých algoritmů pro diskrétní prostor a pro řešení problému obchodního cestujícího popsaných v \ref{section:redefiniceTSP}.

\section{Warehouse Manager}
\label{navrhKonfigurator}
Aplikace bude založena na grafickém frameworku \texttt{Qt} a pro tvorbu modelu budou využity grafické prvky, které budou zasazovány do grafické scény. Dále bude grafické rozhraní doplněno o veškeré nástroje popsané výše, pro jednoduchost použití a pro konzistentnost.
