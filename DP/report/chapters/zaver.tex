Práce měla za úkol vytvořit nástroj, který bude schopen optimalizovat fungování skladu za účelem zvýšení jeho propustnosti. Důležitou podmínkou byla nezávislost optimalizace na modelu skladu (tj. uživatel jej může vytvořit dle svých potřeb), čehož bylo dosaženo za pomoci grafického editoru a realistické simulace. Dále bylo třeba vytvořit generátor syntetických objednávek kvůli citlivosti zákaznických dat a nakonec kvantitativně vyhodnotit dosažené výsledky.

Nástroj pathfinder dokáže nalézt optimální cestu skrze sklad v relativně malém počtu iterací algoritmu a zrychlení zpracování objednávek je téměř dvojnásobné -- $\textbf{59.4\%}$. Stejně tak optimalizátor rozložení produktů dokáže téměř dvojnásobně zrychlit zpracování všech objednávek -- $\textbf{57\%}$, avšak doba pro natrénování je zde značně delší. Kombinací těchto dvou přístupů zároveň pak lze dosáhnout ještě lepších výsledků, avšak za cenu velmi dlouhé optimalizace. Při kombinatorickém nárůstu možných řešení a při zachování nastavení optimalizátoru a délky optimalizace se kvalita optimalizace rozložení snižuje -- nejvýše však~o~$\textbf{10\%}$.

Přínosem této práce je úplný grafický nástroj, jenž dosahuje velmi dobrých výsledků a poskytuje mnoho užitečných funkcí v oblasti skladového hospodářství. Dále tato práce přináší novou metodu k řešení problematiky SLAP a sice kombinaci dvou \emph{state of the art} technik a nakonec přináší nové optimalizační kritérium v kontextu SLAP -- celkovou dobu zpracování sady objednávek.

V budoucnu by bylo možné rozšířit práci o nástroj schopný generovat optimální rozložení skladu na základě uživatelem definovaných podmínek, a to za pomoci CGP (kartézského genetického programování).