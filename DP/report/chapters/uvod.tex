% Motivace
Sklady jsou klíčová část dodavatelského řetězce, kam jsou dočasně uloženy produkty z~výroby, než jsou odeslány k~zákazníkům~v~rámci objednávek. Simulace, automatizace a optimalizace skladů se vzhledem k rostoucím nárokům na jejich výkonnost a propustnost v~posledních letech značně rozšířila. Automatizace a optimalizace skladu, ač velmi nákladný proces, může společnosti přinést nebývalé zvýšení produktivity, bezpečnosti a v neposlední řadě také kvality, respektive menší chybovosti.

% Problém
Sklady se zpravidla optimalizují za účelem rychlejšího odbavování zákaznických objednávek. Každá objednávka sestává z několika položek, kde každá položka jednoznačně identifikuje zakoupený produkt a jeho požadované množství. Objednávky jsou typicky vyřizovány ve formě kartonu, do kterého se vloží zakoupené produkty (popřípadě faktura, atp.) a karton se odešle k zákazníkovi. Vyřizování objednávek v rámci skladu poté sestává z cestování kartonu mezi lokacemi po dopravnících a vybírání požadovaných produktů ze slotů lokací (úložné prostory, kde jsou produkty dočasně uloženy) do kartonů objednávek. Proces sbírání zakoupených produktů do kartonů je obecně označován jako pickování objednávek. Plynulost a efektivita pickování objednávek, které jsou ovlivněny mimo jiné rozložením produktů, mají zásadní vliv na výkonnost skladu jako celku, a proto jsou často považovány za nejslibnější oblast z hlediska optimalizace skladových operací~\cite{optimisationOrderPickingGA}. Tato práce řeší primárně kombinatorický NP-těžký problém, a sice jakým způsobem rozmístit produkty do slotů lokací ve skladu, aby bylo dosaženo co nejvyšší propustnosti. Tato problematika se v literatuře označuje jako SLAP (\emph{storage location assignment problem}).

% Existující práce
Z množství vědeckých příspěvků, které se v posledních letech zabývaly problematikou SLAP lze usuzovat, že je to velmi aktivní a diskutované téma. Z obsahu těchto příspěvků pak lze usuzovat, že toto téma není zdaleka vyřešené, a existuje zde velký potenciál pro možná zlepšení, a to zejména z pohledu zvýšení výkonnosti skladů.

% Způsob řešení
V navrženém řešení uživatel nejprve provede vytvoření modelu skladu v grafickém 2D editoru dle jeho potřeb. Poté si vygeneruje data pomocí \textbf{generátoru objednávek}, který na základě pravděpodobnostních modelů generuje sady zákaznických objednávek pro trénování a testování. Na základě modelu skladu a vygenerovaných objednávek může následně uživatel provést realistickou \textbf{simulaci skladu}, nebo pomocí nástroje \textbf{hledač cest} (dále jako pathfinder) nalézt optimální cestu objednávky skladem pomocí mravenčího algoritmu. Dále je uživatel schopen provést \textbf{optimalizaci rozložení produktů} ve skladu za účelem zvýšení propustnosti skladu pomocí čtyř evolučních algoritmů. Veškeré zmíněné funkcionality reprezentované jednotlivými nástroji jsou jednak použitelné samostatně skrze příkazovou řádku, ale také spojeny do jednoho grafického nástroje, nazvaného \textbf{skladový manažer} (dále jako Warehouse Manager). Hlavním přínosem tedy bude zvýšení propustnosti skladu, a to díky hned několika optimalizačním technikám, které jsou v dokumentu porovnány.

% Zhodnocení (prodání)
Vytvořené řešení je zcela nezávislé na modelu skladu, ten si lze vytvořit zcela libovolně, narozdíl od velké části existujících řešení. Výsledky optimalizace jsou u menších až středních skladů na velmi vysoké úrovni, ačkoli se zvyšující se komplexitou skladu se kvalita optimalizace lehce snižuje. Mimo optimalizaci rozložení produktů ve skladu poskytuje řešení další užitečné funkcionality, jako je například identifikace úzkých míst (tzv. \emph{bottlenecků}) či nalezení nejkratší cesty objednávky skrze sklad.

% Struktura dokumentu
Kapitola \ref{SLAP} se zabývá obecným popisem problematiky alokace produktů do lokací a~souhrnem existujících přístupů. V kapitole \ref{EA} lze nalézt popis evolučních algoritmů a jejich redefinici pro diskrétní prostor. Kapitoly \ref{navrhNastroju} a~\ref{implementaceNastroju} popisují návrh a implementaci pěti nástrojů vytvořených v rámci této práce. Kapitola \ref{vyhodnoceniNastroju} poté shrnuje dosažené výsledky práce.
